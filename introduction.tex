\section{Introduction}\label{intro}\sloppy
Images and video are the frontier of modern data management.
The potential scale is enormous with hundreds of hours of video uploaded to YouTube every minute, the immense data generated from CCTV footage, and a possible future with data collected from autonomous driving. 
Recent advances in deep learning for computer vision have greatly increased the scope of \emph{visual analytics}, where the visual content of the images and videos can be analyzed.
In the last two years, there are a handful of systems that have explored scaling visual analytics pipelines to larger and/or faster streams of data~\cite{anderson2018predicate, kang2018blazeit,kang2017noscope, wu2018querying, sparks2017keystoneml}.

An archetypal visual analytics task is to find all images in a corpus that contain a certain object, e.g., detecting vehicles in traffic camera data.
The proliferation of neural network-based object detection models have made such tasks feasible in the last two years~\cite{he2017mask}.
These models not only identify if an image contains a certain object but also return a segmentation mask (a subimage cropped around the boundaries of all instances of the object in the image).
Object detection models can be pre-trained on vast corpora of offline annotated data\footnote{For example, http://cocodataset.org/} and then can be deployed on unseen images or video frames to find those that have the desired object.

It is worth noting the analogies between this visual analytics pipeline and traditional data analysis methodology.
The raw images and video data are similar to unstructured data, and the object detection model is like a data extraction algorithm that returns a structured output (segmentations and discrete object classes).
Over this structured output, we can filter for those images that contain the object of interest.
In other words, there is an ``ETL'' phase where one or more models is used to construct structured outputs and then there is a ``query'' phase where queries are executed over those structured outputs.
We found this design pattern to be suprisingly general across a number of visual analytics tasks of interest.
This paper explores whether we can make this analogy explicit in the programming model.
We argue that such a relational ``visual'' data model can greatly simplify both the user-facing and system-facing programming interface to scale visual analytics to more complex tasks.

To understand why this might be interesting, consider a variant of the query before: \emph{given two videos find all objects that appear in both videos}.
Answering such a query is not simply a question of evaluating a filter along each stream of images.
One has to find all the objects in both videos and then match them against each other.
Efficiently answering this sort of query raises questions about physical design (should one build an index of frames with the object), query optimization (which video to scan and which index to use), and compression (can the query be answered from features instead of raw frames).
In the design of traditional relational databases, the data and query model allows us to separate physical and logical concerns--abstracting many of these questions from the end user.
We would like to offer the same for visual analytics tasks, where low-level concerns are abstracted away. 

While a true Visual Data Management System (VDMS) is the ultimate goal, we found there are still several key gaps in understanding the tradeoffs and performance of such systems. 
We start with a simple query model, where visual analytics queries are relational queries over collections of subimages (called patches) and associated metadata about how those subimages were generated.
We built a research prototype system, called \textsf{DeepLens}, to implement this query model and a number of additional features such as an ETL layer, storage layer, indexing, and query processing. 
We evaluated this system on a benchmark of 6 visual analytics tasks over 3 datasets and use the experiments to elicit some of the key open research challenges.
To summarize our key lessons learned:

\vspace{0.75em} \noindent \emph{Lesson 1. Explosion of physical design choices. } A successful VDMS will have to manage both single-dimensional indexes (e.g., B+ Trees) and geometric indices (e.g., KD-Trees). We found that index usage was critical and could improve performance by up-to 612x, especially on queries that compared two collections of pixel data to each other.
Particularly for the geometric indices, the size and dimensionality of the data plays a key role in what index to use. Changes to the extraction pipeline, such as new type of image featurization, may lead to substantial and unpredictable changes in index performance.
Automated tools for physical design such as~\cite{sharma2018case,pavlo2017self} will be crucial.

\vspace{0.5em} \noindent \emph{Lesson 2. Almost all processes are compute-limited. }
Image processing algorithms are very compute intensive. This differs from classical DBMS models where IO costs are dominant. 
Many classical indexing, query processing, and query optimization frameworks are designed with IO-limited models in mind.
In a compute-limited world, there is a new space of optimizations that can be explored.
For example, our experiments found that creating disk-based indexes on-the-fly could substantially improve performance since the benfit of reducing comparisons downstream significantly outweighs the I/O overhead.
On the other hand, we found that join costs in image matching problems were unpredictable, non-linear, and sensitive to the particular chipset (GPU, Vectorization, Vanilla CPU).
This poses a challenge for cost-based query optimization and envision that new results in learning-based query optimization~\cite{kaftan2018cuttlefish,krishnan2018deeprljoins} will be a key part of any VDMS.

\vspace{0.5em} \noindent \emph{Lesson 3. Native support for lineage queries. } Semantics from images have to be first extracted with computer vision algorithms before structured queries can be executed. We found that many VA queries can be formulated as different types of joins and filters on the outputs of these algorithms and relating those results back to the base data. For example, finding a set of segmentation masks that satisfy a certain criteria and then overlaying them on the original image.
\textsf{DeepLens} natively maintains tuple-level lineage to the original image or video after any extraction or transformation. This lineage is stored in the datastore and can be indexed and queried like any data--leading to a 60x improvement in one benchmark query.
This idea is in the spirit of recent work in lineage management system such as~\cite{psallidas2018smoke}.


\vspace{0.75em}

While it is natural to connect recent interest in visual analytics with prior work on multimedia databases~\cite{yoshitaka1999survey}, the recent instantiations of visual analytics systems only loosely borrow from past architectural designs.
We argue that as the scope of  visual analytics increases, scaling requires leveraging classical ideas from relational data management--as seen in the past with multimedia databases.
However, the tasks that we evaluate are significantly higher dimensional (images featurized by 1000 of dimensions) than ones considered in the past either in Geospatial databases or Multimedia databases.
Furthermore, the use cases have evolved where the metadata around the images is mostly generated by an automated deep learning pipeline rather than human provided.
Maintaining provenance to the raw data is crucial for debuging and merging different results. 
These differences aside, the core idea is the same: a relational database which holds structured metadata and image data with a relational data and query model. 
This way we can leverage the principles of design from relational database management systems and apply them to VDMS.










