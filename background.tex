\section{Background and API}
Recent work in visual analytics has mostly focused on optimizing the throughput of predictions on a single stream of images~\cite{anderson2018predicate, kang2018blazeit,kang2017noscope, wu2018querying, sparks2017keystoneml}.
However, as a direct consequence of the rapid progress in computer vision, visual analytics workloads are becoming increasingly complex with requirements beyond this paradigm. 
For example, it is not uncommon for visual perception pipelines in robotics to integrate predictions from multiple neural networks~\cite{hodson2018robots}.

We were interested in how practitioners manage and merge predictions from these increasingly complex pipelines. 
We found that existing pipelines are carefully designed and often not robust to changes in image resolution. They can make significant assumptions about ordering of images and the structure of any associated metadata (e.g., adding a new prediction label).
This echoes the challenges of the pre-relational days of database research before the idea of data independence where the query processing, the data, and the storage format are all intertwined.
We want a unified model that substantially covers much of visual analytics that can allow us to leverage logical-physical separation principles that the database community has pioneered.

\subsection{Patch Query Model}
After surveying a number of use cases from robotics to traffic camera analysis, we arrived at a simple query model, where visual analytics queries are relational queries over collections of subimages (called patches) and associated metadata about how those subimages were generated.
The query processing engine is agnostic to how those patches are generated.
They can be whole images, smaller tiled subimages, or even subimages extracted by an object detection neural networks.
In pseudocode, a \texttt{Patch} object contains a pointer to the image that generated it, the pixels contained in the patch, and a key-value metadata dictionary:
\begin{lstlisting}
Patch(ImgRef, Data, MetaData)
\end{lstlisting}
Inspired by dataflow systems, operators in the system implement iterators over tuples of \texttt{Patch} objects:
\begin{lstlisting}
Operator(Iterator<Tuple<Patch>> in, 
         Iterator<Tuple<Patch>> out)
\end{lstlisting}
Lineage is maintained as every operator is required to update the \texttt{ImgRef} attribute to retain a lineage chain back to the original image.

\subsection{Example 1}
Consider a CCTV feed of a parking lot that collects and stores video.
We want to evaluate the parking lot's utilization so we want to count the number of frames in the video that contain at least one car.
We first run all of the frames through the SSD object detection network~\cite{liu2016ssd}, which returns bounding boxes and labels for common objects detected in the image.
Each of these bounding boxes defines a patch:
\begin{lstlisting}
SSDPatch(Frame, Bbox, 
         {'label': L, 'frameno': F })
\end{lstlisting}
Over these SSD patch objects, the query of interest is well expressed in relational algebra over the metadata dictionary (a filter over labels and an aggregation over frame numbers).

\subsection{Example 2}
Now, suppose we are given two videos from two different cameras, we want to find all cars that appear in both videos.
The first step is exactly the same as the previous example, we break the frames up into \texttt{SSDPatch} objects.
Over the two sets of \texttt{SSDPatch}, we need to compare all of the bounding boxes and return those that are sufficiently similar in terms of image content.
What is different about this query is that it leverages both the metadata and the pixel data in the bounding boxes.

What struck us as interesting about this query is that there are a number of unresolved questions about how to actually process it efficiently.
Naively, one could compare all pairs of bounding boxes and then return those of sufficient image similarity.
Most image matching algorithms use lower dimensional features to match, so another option is to pre-compute the relevant features and build a multidimensional index over one of the sets of \texttt{SSDPatch} objects, e.g., a KD-Tree over a set of color histograms.

Additionally, what if we knew that the target vehicle was red.
It is not clear whether we would want to eagerly apply that predicate. 
The predicate is likely CPU intensive and if the matching step has a low selectivity, it might be beneficial to apply the predicate after matching.
These questions motivated us to design a DBMS-like framework to answer these sorts of queries.









