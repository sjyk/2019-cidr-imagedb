








The main insight of this paper is that visual analytics often 










While it is natural to connect this recent interest with prior work on multimedia databases~\cite{yoshitaka1999survey}, the recent instantiations of VA systems only loosely borrow from past architectural designs. Current types of architecture suffice since the focus has been on relatively simple analytics tasks. 
Answering such queries is mostly bottlenecked by neural network inference.

\red{ae: at some point early should we introduce a system term like VDMS and outline the data model?}





This paper takes the position that a relational data model can greatly simplify the user-facing programming interface to scale VA to more complex tasks.
We revisit the idea of a multimedia database in the era of deep learning. 
This database can load corpora of images, videos and answer semantic queries about the content in these videos based on a library of computer vision algorithms that segment, annotate, and process images.

\red{ae: not clear why relational is the right data model at this point, what parts of it are being borrowed?} 






We present initial experimental results on these components to elicit some of the key open research challenges. 

